\documentclass[a4paper, 11pt]{scrartcl}
\usepackage[utf8]{inputenc}
\usepackage[english]{babel}
\usepackage[T1]{fontenc}
\usepackage{amsmath}
\usepackage{graphicx}
\usepackage{url}
\usepackage{float}
\usepackage{fancyhdr}
\bibliographystyle{plainnat}
\newcommand{\uproman}[1]{\uppercase\expandafter{\romannumeral#1}}

\title {Development and construction of a 2D Laser Scanner}
\author {Niklas H}
\date {\today}

\begin{document}
\maketitle
\tableofcontents
\newpage
\section{Introduction}

\section{Prototype calculations}
%pic
The main assumptions of the Prototype are a operiational voltage of 5V and a maximum laser power of 100mW. The Peltier element can produce a maximum of 3.8V at 8.4W. Assuming linear behaviour \footnote{This is most likely untrue but will suffice for the first calculations.}, a slope of $S = \frac{3.8V}{8.4W}=0.45\frac{V}{W}$ can be calculated. With the maximum laser power of $P_{max}=100mW=0.1W$, the Peltier element is expected to produce a maximum of $S\cdot P_{max} = 0.45\frac{V}{W}\cdot 0.1W = 0.045V$. In order to amplify this signal, an operational amplifier is used (specificially a LM358), with a gain factor of $\beta = \frac{5V}{0.045V} = 111$. Two resistors are used to set this gain, with $R_1$ connecting the inverting pin to ground and $R_2$ connecting the output to the inverting pin (the positive lead of the Peltier element is connected to the noninverting pin). The formula describing the op-amp gain is as follows:
\begin{equation}
111 = 1 + \frac{R_2}{R_1} \Leftrightarrow
110 = \frac{R_2}{R_1}
\end{equation}
With $R_2$ arbitrarily chosen as 10k$\Omega$, $R_1$ is 90$\Omega$, which is rounded to 100$\Omega$. The resulting gain of $1+\frac{10000\Omega}{100\Omega}=101$ is acceptable. \\
\section{Acknowledgements}


%\addcontentsline{toc}{section}{References}
%\bibliography{literature}
\end{document}